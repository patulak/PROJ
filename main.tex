%Aplikace bude obsahovat následující funkce:
%Správa skupin (Oddílů): Možnost vytvářet skupiny, přidávat skauty a vedoucí oddílu
%Události: různé srazy, akce nebo výpomocné činnosti
%Záznam účasti skautů na akcích.
%Komunikace: zasílání oznámení a zpráv členům oddílu.
%Přidělovat "bobříky" skautům za určité úkoly.

\documentclass{article}
\usepackage[czech]{babel}
\usepackage[a4paper,top=2cm,bottom=2cm,left=3cm,right=3cm,marginparwidth=1.75cm]{geometry}
\usepackage{hyperref}
%\usepackage{blindtext}

\title{Správa skautského systému}
\author{Patrik Kolář}
\date{}
\begin{document}
\maketitle
\newpage

\tableofcontents
\newpage

\section{Úvod}
\subsection{Téma a cíl práce}
Cílem práce je vytvořit jednoduchou a přehlednou aplikaci jak pro vedoucí skautských oddílů tak pro mladé skauty. Aplikace by měla sloužit pro administrativní potřeby a komunikaci mezi jednotlivými členy. Vedoucí tak budou moci vytvářet nové události a akce, zaznamenávat účast skautů a udělovat "bobříky".
\subsection{Skauting}
Skauting se poprvé objevil v Anglii v roce 1907, za jeho vznikem stál britský generál Robert Baden-Powell. O pár let později roku 1912 proběhl první skautský tábor i v tehdejším Československu, který vedl Antonín Benjamín Svojsík. Z počátku byl skauting určený pouze pro mladé chlapce, ale brzy se do skautských oddílů dostali i dívky. \\
Skauti byly rozděleni do oddílů po menších počtech a každému oddílu byl navíc přiřazen starší a zkušenější jedinec. Hlavním významem tohoto sdružení byla výchova mládeže, naučit je čestnému chování, úctě k lidem, přírodě a sama sobě. \\
Později byly skauti rozděleni na starší a mladší oddíly pro zjednodušení organizace. Dnes je v česku nejrozšířenější skautská organizace  "Junák – český skaut, z. s.".

\section{Seznámení s problematikou}
\subsection{Organizace}
\subsection{Administrativa}

\section{Analýza existujících aplikací}
\subsection{Vlastnosti a funkce}
\subsection{SkautlS}
\subsection{TroopTrack}

\section{Návrh řešení}
\subsection{Požadavky}
\subsection{Koncept aplikace}
\subsection{Návrh databáze}
\subsection{Bezpečnost}

\section{Technologie}
\subsection{Python}
\subsection{SQL databáze}

\section{Závěr}
\subsection{Shrnutí}
\subsection{Možná vylepšení}

\section{Zdroje}
Bobříky: \url{https://foglarweb.skauting.cz/clanky.php?id=108} \\
Skauting: \url{https://cs.wikipedia.org/wiki/Skauting} \\
          \url{https://www.skaut.cz/skauting/historie/}

\end{document}